\documentclass[11pt]{article}

\input{../../Latex_Common/skinnerr_latex_preamble_asen5417.tex}

%%
%% DOCUMENT START
%%

\begin{document}

\pagestyle{fancyplain}
\lhead{}
\chead{}
\rhead{}
\lfoot{\hrule ASEN 5417: Homework 7}
\cfoot{\hrule \thepage}
\rfoot{\hrule Ryan Skinner}

\noindent
{\Large Homework 7}
\hfill
{\large Ryan Skinner}
\\[0.5ex]
{\large ASEN 5417: Numerical Methods}
\hfill
{\large Due 2015/12/1}\\
\hrule
\vspace{6pt}

%%%%%%%%%%%%%%%%%%%%%%%%%%%%%%%%%%%%%%%%%%%%%%%%%
%%%%%%%%%%%%%%%%%%%%%%%%%%%%%%%%%%%%%%%%%%%%%%%%%
\section{Introduction} %%%%%%%%%%%%%%%%%%%%%%%%%%
%%%%%%%%%%%%%%%%%%%%%%%%%%%%%%%%%%%%%%%%%%%%%%%%%
%%%%%%%%%%%%%%%%%%%%%%%%%%%%%%%%%%%%%%%%%%%%%%%%%

Consider the non-linear, inviscid Burgers equation for $u(x,t)$,
\begin{equation}
\pp{u}{t} + u \pp{u}{x} = 0
\;,
\end{equation}
with the initial conditions
\begin{equation}
\begin{aligned}
u(x,0) &= 10 &\quad 0  \le\; &x \le 30 \;, \\
u(x,0) &= 0  &\quad 30 \le\; &x \le 40 \;.
\end{aligned}
\end{equation}

Use the following finite-difference approximations to numerically integrate this equation using appropriate Dirichlet or Neumann BCs on an x-grid with $\Delta x = 0.2$:
\begin{enumerate}
\item MacCormack explicit method
\item Beam and Warming implicit method
\end{enumerate}

Note that the second method  may require the incorporation of a smoothing operator added directly to the finite difference formula. Using a fourth-order artificial viscosity, optimize the coefficient of this operator for minimum amplitude errors,
\begin{equation}
D_\epsilon = - \epsilon(\Delta x)^4 \pp{^4 V}{x^4}
\;,
\end{equation}
where the negative sign ensures that positive dissipation is produced. Using central differences, we obtain
\begin{equation}
\epsilon(\Delta x)^4 \pp{^4 V}{x^4} = V_{i-2} -4V_{i-1} +6V_i -4V{i+1} +V{i+2}
\;.
\end{equation}
The coefficient $\epsilon$ generally obeys $0 \le \epsilon \le 1/8$, with a preferred value of $\epsilon = 0.1$.

For both methods, plot the solutions at intervals of about two time units up to about $t=8$ time units. Obtain solutions for condition numbers of $CN = \{\tfrac{3}{4}, 1, \tfrac{5}{4}\}$. Comment on the stability of the scheme and dispersive/dissipative errors.

%%%%%%%%%%%%%%%%%%%%%%%%%%%%%%%%%%%%%%%%%%%%%%%%%
%%%%%%%%%%%%%%%%%%%%%%%%%%%%%%%%%%%%%%%%%%%%%%%%%
\section{Methodology} %%%%%%%%%%%%%%%%%%%%%%%%%%%
%%%%%%%%%%%%%%%%%%%%%%%%%%%%%%%%%%%%%%%%%%%%%%%%%
%%%%%%%%%%%%%%%%%%%%%%%%%%%%%%%%%%%%%%%%%%%%%%%%%

%%%%%%%%%%%%%%%%%%%%%%%%%%%%%%%%%%%%%%%%%%%%%%%%%
%%%%%%%%%%%%%%%%%%%%%%%%%%%%%%%%%%%%%%%%%%%%%%%%%
\section{Results} %%%%%%%%%%%%%%%%%%%%%%%%%%%%%%%
%%%%%%%%%%%%%%%%%%%%%%%%%%%%%%%%%%%%%%%%%%%%%%%%%
%%%%%%%%%%%%%%%%%%%%%%%%%%%%%%%%%%%%%%%%%%%%%%%%%

%\begin{figure}[h!]
%\begin{center}
%\includegraphics[scale=0.6]{Prob2_convergence.eps}
%\includegraphics[scale=0.6]{Prob2_contours.eps}
%\\[0.5cm]
%\caption{Convergence behavior and contours of $U$ for Problem 2. Neighboring values of $\omega = 1.937$ and $\omega = 1.940$ both converge in 202 iterations.}
%\label{fig:Prob2}
%\end{center}
%\end{figure}

%%%%%%%%%%%%%%%%%%%%%%%%%%%%%%%%%%%%%%%%%%%%%%%%%
%%%%%%%%%%%%%%%%%%%%%%%%%%%%%%%%%%%%%%%%%%%%%%%%%
\section{Discussion} %%%%%%%%%%%%%%%%%%%%%%%%%%%%
%%%%%%%%%%%%%%%%%%%%%%%%%%%%%%%%%%%%%%%%%%%%%%%%%
%%%%%%%%%%%%%%%%%%%%%%%%%%%%%%%%%%%%%%%%%%%%%%%%%

%%%%%%%%%%%%%%%%%%%%%%%%%%%%%%%%%%%%%%%%%%%%%%%%%
%%%%%%%%%%%%%%%%%%%%%%%%%%%%%%%%%%%%%%%%%%%%%%%%%
\section{References} %%%%%%%%%%%%%%%%%%%%%%%%%%%%
%%%%%%%%%%%%%%%%%%%%%%%%%%%%%%%%%%%%%%%%%%%%%%%%%
%%%%%%%%%%%%%%%%%%%%%%%%%%%%%%%%%%%%%%%%%%%%%%%%%

No external references were used other than the course notes for this assignment.

%%%%%%%%%%%%%%%%%%%%%%%%%%%%%%%%%%%%%%%%%%%%%%%%%
%%%%%%%%%%%%%%%%%%%%%%%%%%%%%%%%%%%%%%%%%%%%%%%%%
\section*{Appendix: MATLAB Code} %%%%%%%%%%%%%%%%
%%%%%%%%%%%%%%%%%%%%%%%%%%%%%%%%%%%%%%%%%%%%%%%%%
%%%%%%%%%%%%%%%%%%%%%%%%%%%%%%%%%%%%%%%%%%%%%%%%%

The following code listings generate all figures presented in this homework assignment.

%\includecode{Problem_1.m}

%%
%% DOCUMENT END
%%
\end{document}
